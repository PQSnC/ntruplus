\documentclass[11pt, letterpaper]{article}
\usepackage{times, mathptmx}
\DeclareMathAlphabet{\mathcal}{OMS}{cmsy}{m}{n}
\usepackage{fullpage}
\usepackage{amsmath, amsfonts, amssymb, amsthm, amscd}
\usepackage{hyperref, cite, url}

\newcommand{\vs}{\vspace{1.5mm}}

% theorem
\theoremstyle{definition}
\newtheorem{theorem}{Theorem}[section]
\newtheorem{lemma}[theorem]{Lemma}
\newtheorem{definition}[theorem]{Definition}


\title{NTRU+}

\author{ Jonghyun Kim \footnote{Korea University, Seoul, Korea.
        Email:\texttt{yoswuk@korea.ac.kr}.}
    \and Jong Hwan Park\footnote{Institution2. Email: \texttt{abc@abc.abc}.} }


\begin{document}
\maketitle
\date{}

\begin{abstract} In this section, ..
\end{abstract}

\vs \noindent {\bf Keywords:} Post Quantum Signature, Pseudorandom Generator

\section{Introduction}

\begin{definition}[RLWE problem]
    Let $n$, $q$, $p$ be positive integers such that $q > p$. Let $\mathcal{R}_q$
    and $\mathcal{R}_p$ be polynomial rings constructed by $\Phi_{3n}(x)$, and
    $\mathcal{D}_s$ be a distribution over $\mathcal{R}_q$. A decisional RLWR
    $problem \mathsf{RLWR}_{n,1,q,p}(\Phi_{3n})$ is to distinguish uniformly
    random $(a, u) \in \mathcal{R}_q \times \mathcal{R}_p$ and
\end{definition}

\begin{comment}
    $(a, b = \round{\frac{p}{q} a \cdot s}) \in \mathcal{R}_q

    \times \mathcal{R}_p$ where $\vc{s}$ is sampled from $\mathcal{D}_s$. Then,
    the advantage of an adversary $\mathcal{A}$ in solving the decisional RLWR
    problem $\mathsf{RLWR}_{n,1,q,p}(\mathcal{D}_s)$ is defined as follows:
    \begin{center}
        $\Adv^\mathcal{RLWR}_{n,1,q,p}(\mathcal{A}) =  \Pr[\mathcal{A}(a, b) = 1] - \Pr[\mathcal{A}(a, u) = 1]$.
    \end{center}   
\end{comment}



\section{Base Encryption Scheme}
\begin{description}
    \item $\textbf{KeyGen}$:
    \begin{itemize}
        \item $f' \leftarrow \mathcal{R}$
        \item $f = 3f'+1$
        \item if $f$ is not invertible in $R_q$, restart
        \item $g \leftarrow \mathcal{R}$
        \item $h = 3g/f$
        \item $\textbf{return} (sk=f,pk=h)$
    \end{itemize}
    \item $\textbf{Encrypt}$:
    \begin{itemize}
        \item $c = hr + m$
    \end{itemize}
    \item $\textbf{Decrypt}$:
    \begin{itemize}
        \item $m = (cf \mod^{\pm} q) \mod^{\pm} 3$
    \end{itemize}
\end{description}

\end{document}
%%%%%%%%%%%%%%%%%%%%%%%%%%%%%%%%%%%%%%%%%%%%%%%%%%%%%%%%%%%%%%%%%%%%%%%%%%%%%